\documentclass[12pt, a4paper]{article}
\usepackage[brazil]{babel}

% header
\usepackage{fancyhdr}
% docx file had: header = 1.75cm; top = 1.25cm; image = (floating) 2cm
% so the LaTeX top = 1.75cm + 1.25cm = 3cm

% docx file had: footer = 1.29cm; bottom = 0.47cm
% so the LaTeX bottom = 1.76cm
\usepackage[left=3cm,top=3cm,right=2.5cm,bottom=1.76cm]{geometry}
\pagestyle{fancy}

\usepackage{tikz}
\usetikzlibrary{calc}
\usepackage{tikzpagenodes}

\renewcommand{\headrulewidth}{0pt}

\fancyhf{} % removes section name from header
\lhead{
	\begin{tikzpicture}[remember picture,overlay]
		\draw  let \p1=($(current page.north)-(current page header 
		area.south)$),
		\n1={veclen(\x1,\y1)} in
		node [inner sep=0,outer sep=0,below right] 
		at (current page.north 
		west){\includegraphics[width=21cm,height=2cm]{images/cit.png}};
	\end{tikzpicture}
}

% sections centering fontsize and uppercase
\usepackage{sectsty}
\sectionfont{\centering\fontsize{12}{15}\selectfont\MakeUppercase}

% sections 1.(space)name
\makeatletter % access to internal commands
\renewcommand{\@seccntformat}[1]{\csname the#1\endcsname.\ }
\makeatother

% spacing before and after sections
\usepackage{titlesec}
\titlespacing*{\section}
{0pt}{12pt}{12pt}

% line spacing
\usepackage{setspace}
\singlespacing 

% paragraph
%\usepackage{parskip}
\setlength{\parindent}{0.97cm}
\usepackage{indentfirst} % indents the first paragraph after a section¨

% font
\usepackage{fontspec} % fonts
\setmainfont{Arial} % sets the font

% links
\usepackage{hyperref}
\urlstyle{same} % sets the font of \url to the same as the rest of the document
 % you can change the header (event name) inside the 
%preamble file

\begin{document}
	\begin{center}
		\textbf{\uppercase{TÍTULO DO TRABALHO}}
		\vskip 12pt
		\underline{\uppercase{NOME E SOBRENOME DO AUTOR}$^1$}; \uppercase{NOME E SOBRENOME DO(S) 
		CO-AUTOR(ES)}$^2$; \uppercase{NOME E SOBRENOME DO ORIENTADOR}$^3$
		\vskip 12pt
		\begin{footnotesize}
			\textit{
				$^1$Nome da Instituição do Autor 1 – e-mail do autor 1
			}
		
			\textit{
				$^2$Nome da Instituição do(s) Co-Autor(es) – e-mail do autor 2 
				(se houver)
			}
		
			\textit{
				$^3$Nome da Instituição do Orientador – e-mail do orientador
			}
		
		\end{footnotesize}
		
	\end{center}
	\vskip 12pt
	
	\section{INTRODUÇÃO}
	Este espaço se destina a apresentação do tema do trabalho. O autor deve se 
	preocupar em deixar evidente o assunto que será tratado, a área do 
	conhecimento na qual o trabalho é realizado e apresentar a problematização 
	que especifica o seu estudo. 

	A fundamentação teórica do trabalho é uma parte importante da introdução, 
	onde o autor deverá explicitar as fontes bibliográficas e o entendimento 
	que existe sobre o tema trabalhado. Também é na introdução que o autor deve 
	expor os objetivos do trabalho.
	
	As citações das referências bibliográficas deverão ser feitas com letras 
	maiúsculas, seguidas do ano de publicação, conforme exemplos: “Esses 
	resultados estão de acordo com os reportados por MILLER; JUNGER (2010) e 
	LEE et al. (2011), como uma má formação congênita (MARTINS, 2005)”.	
	
	O corpo do texto do resumo deve estar em fonte Arial, corpo 12. Os títulos 
	de seções devem estar centralizados, com letra maiúscula, em negrito e em 
	fonte Arial, corpo 12.
	
	\section{METODOLOGIA}
	Aqui o autor deve explicar como o trabalho foi realizado, expondo os 	
	procedimentos que foram adotados para a realização da pesquisa e geração 
	dos resultados. A fundamentação metodológica deve esclarecer os trabalhos 
	que embasam a análise proposta.
	
	\section{RESULTADOS E DISCUSSÃO}
	A preocupação nesta parte do resumo deve ser a de expor o que já foi feito 
	até o momento, quais os resultados encontrados e o estado em que se 
	encontra o trabalho. Esta parte serve também para que o autor evidencie o 
	desenvolvimento do trabalho, ou seja, a análise do trabalho de campo e do	
	objeto de estudo propriamente dito.
	
	Se forem usadas tabelas e figuras, seus títulos deverão ser centralizados, 
	com as letras iniciais maiúsculas e fonte Arial, corpo 12. 
	
	%\vskip 24pt the official docx templates leaves 2 12pt blank lines here
	\section{CONCLUSÕES}
	Nas conclusões o autor deve apresentar objetivamente qual a inovação obtida 
	com o trabalho, evitando apresentar resultados neste espaço. 

	\section{REFERÊNCIAS BIBLIOGRÁFICAS}
	\noindent\underline{Livro}
	\\
	SOBRENOME, Letras Iniciais dos Nomes. \textbf{Título do Livro}. Local de 
	Edição: Editora, ano da publicação. Ex.: JENNINGS, P.B. \textbf{The 
	practice of large animal surgery}. Philadelphia: Saunders, 1985. 2v.
	%\\\\\\ the official docx template leaves 2 12pt blank lines here\\
	\\\\
	\underline{Capítulo de livro}
	\\
	SOBRENOME, Letras Iniciais dos Nomes (do autor do capítulo). Título do 
	capítulo. In: SOBRENOME, Letras Iniciais dos Nomes (Ed., Org., Comp.) 	
	\textbf{Título do Livro}. Local de Edição: Editora, ano de publicação.	
	Número do Capítulo, p. página inicial – página final do capítulo.	Ex.: 
	GORBAMAN, A.A. comparative pathology of thyroid. In: HAZARD, J.B.; SMITH, 
	D.E. \textbf{The thyroid}. Baltimore: Williams \& Wilkins, 1964. Cap.2, 
	p.32-48.
	\\\\
	\underline{Artigo}
	\\
	SOBRENOME, Letras Iniciais dos Nomes. Título do Artigo. \textbf{Nome da 	
	Revista}, 	Local de Edição, v.?, n.?, p. página inicial - página final, 
	ano da 	publicação.	Ex.: MEWIS, I.; ULRICHS, C.H. Action of amorphous 
	diatomaceous earth 	against different stages of the stored product pests 
	Tribolium	confusum(Coleoptera: Tenebrionidae), Tenebrio molitor 	
	(Coleoptera:Tenebrionidae), Sitophilus granarius (Coleoptera: 	
	Curculionidae) and Plodia interpunctella (Lepidoptera: Pyralidae).	
	\textbf{Journal of Stored Product Research, Amsterdam}, v.37, n.1, 
	p.153-164, 2001. 
	\\\\
	\underline{Tese/Dissertação/Monografia}
	\\
	SOBRENOME, Letras Iniciais dos Nomes. \textbf{Título da	
	tese/dissertação/monografia.} Data de publicação. 	
	Tese/Dissertação/monografia (Doutorado/Mestrado/Especialização em ...) - 	
	Programa, Universidade.	Ex.: KLEINOWSKI, A.M. \textbf{Produção de 
	betacianina, crescimento e 	potencial 	bioativo de plantas do gênero 
	Alternanthera}. 2011. 71f. Dissertação 	(Mestrado em Fisiologia Vegetal) 
	- Curso de Pós-graduação em Fisiologia 	Vegetal, Universidade Federal de 
	Pelotas.
	\\\\
	\underline{Resumo de Evento}
	\\
	SOBRENOME, Letras Iniciais dos Nomes. Título do trabalho. In: \textbf{NOME 
	DO EVENTO EM CAIXA ALTA}, 5., Cidade, ano. Título Anais, Proceedings... 
	Local de edição: Editora, ano. página do trabalho.	Ex.: RIZZARDI, M.A.; 
	MILGIORANÇA, M.E. Avaliação de cultivares do ensaio 	nacional de 
	girassol. In: \textbf{JORNADA DE PESQUISA DA UFSM}, 1., Santa Maria, 1992, 
	\textbf{Anais...} Santa Maria: Pró-reitoria de Pós-graduação e Pesquisa, 
	1992. v.1. p.420.
	\\\\
	\underline{Documentos eletrônicos}
	\\
	UFRGS. \textbf{Transgênicos}. Zero Hora Digital, Porto Alegre, 23 mar. 
	2000. Especiais. Acessado em 23 mar. 2000. Online. Disponível em: 
	\url{http://www.zh.com.br/especial/index.htm}
\end{document}
